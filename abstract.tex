%% The following is a directive for TeXShop to indicate the main file
%%!TEX root = diss.tex

\chapter{Abstract}

Recent improvements to deep generative models (DGMs) are leading to increasingly faithful models of real-world data including images and videos. Of particular practical interest are conditional DGMs, which parameterise a conditional probability distribution. This means that they can take an input, such as the first frame of a video, and be used to sample possible data, such as the rest of the video frames, conditioned on the input. Conditional DGMs have many use-cases but are expensive to train and must be re-trained for each new task of interest. This is an issue if, for example, we had the aforementioned model that conditions on a first frame but were then told that we needed to sample videos conditioned on the \textit{final} frame. We posit that many modern DGMs can be altered to avoid this limitation. To show this, we propose methods for training DGMs that are compatible with many different tasks at test-time without retraining. We call such models \textit{flexible} DGMs. We can use the same flexible DGM to sample a video conditioned on any of the first frame, the last frame, both of these, or any other set of frames. We show that flexible DGMs can be designed to have performance similar to, or sometimes better than, standard conditional DGMs. We will demonstrate that the diffusion model class of DGMs is often naturally amenable to being used for flexible conditioning in this way before presenting innovations to enable flexible conditioning in cases where it is not straightforward: when operating under memory constraints with complex data; and when the data has varying and unknown dimensionality that may depend on what we condition on. We conclude by showing that this amenability to flexible conditioning is not unique to diffusion models via a demonstration of a flexible variational auto-encoder. On tasks including image inpainting and video generation, the artefacts we present demonstrate qualitatively and quantitatively improved generation quality versus baselines as well as enabling controllability through flexible conditioning.
